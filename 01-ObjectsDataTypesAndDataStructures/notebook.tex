
% Default to the notebook output style

    


% Inherit from the specified cell style.




    
\documentclass[11pt]{article}

    
    
    \usepackage[T1]{fontenc}
    % Nicer default font (+ math font) than Computer Modern for most use cases
    \usepackage{mathpazo}

    % Basic figure setup, for now with no caption control since it's done
    % automatically by Pandoc (which extracts ![](path) syntax from Markdown).
    \usepackage{graphicx}
    % We will generate all images so they have a width \maxwidth. This means
    % that they will get their normal width if they fit onto the page, but
    % are scaled down if they would overflow the margins.
    \makeatletter
    \def\maxwidth{\ifdim\Gin@nat@width>\linewidth\linewidth
    \else\Gin@nat@width\fi}
    \makeatother
    \let\Oldincludegraphics\includegraphics
    % Set max figure width to be 80% of text width, for now hardcoded.
    \renewcommand{\includegraphics}[1]{\Oldincludegraphics[width=.8\maxwidth]{#1}}
    % Ensure that by default, figures have no caption (until we provide a
    % proper Figure object with a Caption API and a way to capture that
    % in the conversion process - todo).
    \usepackage{caption}
    \DeclareCaptionLabelFormat{nolabel}{}
    \captionsetup{labelformat=nolabel}

    \usepackage{adjustbox} % Used to constrain images to a maximum size 
    \usepackage{xcolor} % Allow colors to be defined
    \usepackage{enumerate} % Needed for markdown enumerations to work
    \usepackage{geometry} % Used to adjust the document margins
    \usepackage{amsmath} % Equations
    \usepackage{amssymb} % Equations
    \usepackage{textcomp} % defines textquotesingle
    % Hack from http://tex.stackexchange.com/a/47451/13684:
    \AtBeginDocument{%
        \def\PYZsq{\textquotesingle}% Upright quotes in Pygmentized code
    }
    \usepackage{upquote} % Upright quotes for verbatim code
    \usepackage{eurosym} % defines \euro
    \usepackage[mathletters]{ucs} % Extended unicode (utf-8) support
    \usepackage[utf8x]{inputenc} % Allow utf-8 characters in the tex document
    \usepackage{fancyvrb} % verbatim replacement that allows latex
    \usepackage{grffile} % extends the file name processing of package graphics 
                         % to support a larger range 
    % The hyperref package gives us a pdf with properly built
    % internal navigation ('pdf bookmarks' for the table of contents,
    % internal cross-reference links, web links for URLs, etc.)
    \usepackage{hyperref}
    \usepackage{longtable} % longtable support required by pandoc >1.10
    \usepackage{booktabs}  % table support for pandoc > 1.12.2
    \usepackage[inline]{enumitem} % IRkernel/repr support (it uses the enumerate* environment)
    \usepackage[normalem]{ulem} % ulem is needed to support strikethroughs (\sout)
                                % normalem makes italics be italics, not underlines
    

    
    
    % Colors for the hyperref package
    \definecolor{urlcolor}{rgb}{0,.145,.698}
    \definecolor{linkcolor}{rgb}{.71,0.21,0.01}
    \definecolor{citecolor}{rgb}{.12,.54,.11}

    % ANSI colors
    \definecolor{ansi-black}{HTML}{3E424D}
    \definecolor{ansi-black-intense}{HTML}{282C36}
    \definecolor{ansi-red}{HTML}{E75C58}
    \definecolor{ansi-red-intense}{HTML}{B22B31}
    \definecolor{ansi-green}{HTML}{00A250}
    \definecolor{ansi-green-intense}{HTML}{007427}
    \definecolor{ansi-yellow}{HTML}{DDB62B}
    \definecolor{ansi-yellow-intense}{HTML}{B27D12}
    \definecolor{ansi-blue}{HTML}{208FFB}
    \definecolor{ansi-blue-intense}{HTML}{0065CA}
    \definecolor{ansi-magenta}{HTML}{D160C4}
    \definecolor{ansi-magenta-intense}{HTML}{A03196}
    \definecolor{ansi-cyan}{HTML}{60C6C8}
    \definecolor{ansi-cyan-intense}{HTML}{258F8F}
    \definecolor{ansi-white}{HTML}{C5C1B4}
    \definecolor{ansi-white-intense}{HTML}{A1A6B2}

    % commands and environments needed by pandoc snippets
    % extracted from the output of `pandoc -s`
    \providecommand{\tightlist}{%
      \setlength{\itemsep}{0pt}\setlength{\parskip}{0pt}}
    \DefineVerbatimEnvironment{Highlighting}{Verbatim}{commandchars=\\\{\}}
    % Add ',fontsize=\small' for more characters per line
    \newenvironment{Shaded}{}{}
    \newcommand{\KeywordTok}[1]{\textcolor[rgb]{0.00,0.44,0.13}{\textbf{{#1}}}}
    \newcommand{\DataTypeTok}[1]{\textcolor[rgb]{0.56,0.13,0.00}{{#1}}}
    \newcommand{\DecValTok}[1]{\textcolor[rgb]{0.25,0.63,0.44}{{#1}}}
    \newcommand{\BaseNTok}[1]{\textcolor[rgb]{0.25,0.63,0.44}{{#1}}}
    \newcommand{\FloatTok}[1]{\textcolor[rgb]{0.25,0.63,0.44}{{#1}}}
    \newcommand{\CharTok}[1]{\textcolor[rgb]{0.25,0.44,0.63}{{#1}}}
    \newcommand{\StringTok}[1]{\textcolor[rgb]{0.25,0.44,0.63}{{#1}}}
    \newcommand{\CommentTok}[1]{\textcolor[rgb]{0.38,0.63,0.69}{\textit{{#1}}}}
    \newcommand{\OtherTok}[1]{\textcolor[rgb]{0.00,0.44,0.13}{{#1}}}
    \newcommand{\AlertTok}[1]{\textcolor[rgb]{1.00,0.00,0.00}{\textbf{{#1}}}}
    \newcommand{\FunctionTok}[1]{\textcolor[rgb]{0.02,0.16,0.49}{{#1}}}
    \newcommand{\RegionMarkerTok}[1]{{#1}}
    \newcommand{\ErrorTok}[1]{\textcolor[rgb]{1.00,0.00,0.00}{\textbf{{#1}}}}
    \newcommand{\NormalTok}[1]{{#1}}
    
    % Additional commands for more recent versions of Pandoc
    \newcommand{\ConstantTok}[1]{\textcolor[rgb]{0.53,0.00,0.00}{{#1}}}
    \newcommand{\SpecialCharTok}[1]{\textcolor[rgb]{0.25,0.44,0.63}{{#1}}}
    \newcommand{\VerbatimStringTok}[1]{\textcolor[rgb]{0.25,0.44,0.63}{{#1}}}
    \newcommand{\SpecialStringTok}[1]{\textcolor[rgb]{0.73,0.40,0.53}{{#1}}}
    \newcommand{\ImportTok}[1]{{#1}}
    \newcommand{\DocumentationTok}[1]{\textcolor[rgb]{0.73,0.13,0.13}{\textit{{#1}}}}
    \newcommand{\AnnotationTok}[1]{\textcolor[rgb]{0.38,0.63,0.69}{\textbf{\textit{{#1}}}}}
    \newcommand{\CommentVarTok}[1]{\textcolor[rgb]{0.38,0.63,0.69}{\textbf{\textit{{#1}}}}}
    \newcommand{\VariableTok}[1]{\textcolor[rgb]{0.10,0.09,0.49}{{#1}}}
    \newcommand{\ControlFlowTok}[1]{\textcolor[rgb]{0.00,0.44,0.13}{\textbf{{#1}}}}
    \newcommand{\OperatorTok}[1]{\textcolor[rgb]{0.40,0.40,0.40}{{#1}}}
    \newcommand{\BuiltInTok}[1]{{#1}}
    \newcommand{\ExtensionTok}[1]{{#1}}
    \newcommand{\PreprocessorTok}[1]{\textcolor[rgb]{0.74,0.48,0.00}{{#1}}}
    \newcommand{\AttributeTok}[1]{\textcolor[rgb]{0.49,0.56,0.16}{{#1}}}
    \newcommand{\InformationTok}[1]{\textcolor[rgb]{0.38,0.63,0.69}{\textbf{\textit{{#1}}}}}
    \newcommand{\WarningTok}[1]{\textcolor[rgb]{0.38,0.63,0.69}{\textbf{\textit{{#1}}}}}
    
    
    % Define a nice break command that doesn't care if a line doesn't already
    % exist.
    \def\br{\hspace*{\fill} \\* }
    % Math Jax compatability definitions
	\def\TeX{\mbox{T\kern-.14em\lower.5ex\hbox{E}\kern-.115em X}}
	\def\LaTeX{\mbox{L\kern-.325em\raise.21em\hbox{$\scriptstyle{A}$}\kern-.17em}\TeX}

    \def\gt{>}
    \def\lt{<}
    % Document parameters
    \title{slides}
    
    
    

    % Pygments definitions
    
\makeatletter
\def\PY@reset{\let\PY@it=\relax \let\PY@bf=\relax%
    \let\PY@ul=\relax \let\PY@tc=\relax%
    \let\PY@bc=\relax \let\PY@ff=\relax}
\def\PY@tok#1{\csname PY@tok@#1\endcsname}
\def\PY@toks#1+{\ifx\relax#1\empty\else%
    \PY@tok{#1}\expandafter\PY@toks\fi}
\def\PY@do#1{\PY@bc{\PY@tc{\PY@ul{%
    \PY@it{\PY@bf{\PY@ff{#1}}}}}}}
\def\PY#1#2{\PY@reset\PY@toks#1+\relax+\PY@do{#2}}

\expandafter\def\csname PY@tok@gd\endcsname{\def\PY@tc##1{\textcolor[rgb]{0.63,0.00,0.00}{##1}}}
\expandafter\def\csname PY@tok@gu\endcsname{\let\PY@bf=\textbf\def\PY@tc##1{\textcolor[rgb]{0.50,0.00,0.50}{##1}}}
\expandafter\def\csname PY@tok@gt\endcsname{\def\PY@tc##1{\textcolor[rgb]{0.00,0.27,0.87}{##1}}}
\expandafter\def\csname PY@tok@gs\endcsname{\let\PY@bf=\textbf}
\expandafter\def\csname PY@tok@gr\endcsname{\def\PY@tc##1{\textcolor[rgb]{1.00,0.00,0.00}{##1}}}
\expandafter\def\csname PY@tok@cm\endcsname{\let\PY@it=\textit\def\PY@tc##1{\textcolor[rgb]{0.25,0.50,0.50}{##1}}}
\expandafter\def\csname PY@tok@vg\endcsname{\def\PY@tc##1{\textcolor[rgb]{0.10,0.09,0.49}{##1}}}
\expandafter\def\csname PY@tok@vi\endcsname{\def\PY@tc##1{\textcolor[rgb]{0.10,0.09,0.49}{##1}}}
\expandafter\def\csname PY@tok@vm\endcsname{\def\PY@tc##1{\textcolor[rgb]{0.10,0.09,0.49}{##1}}}
\expandafter\def\csname PY@tok@mh\endcsname{\def\PY@tc##1{\textcolor[rgb]{0.40,0.40,0.40}{##1}}}
\expandafter\def\csname PY@tok@cs\endcsname{\let\PY@it=\textit\def\PY@tc##1{\textcolor[rgb]{0.25,0.50,0.50}{##1}}}
\expandafter\def\csname PY@tok@ge\endcsname{\let\PY@it=\textit}
\expandafter\def\csname PY@tok@vc\endcsname{\def\PY@tc##1{\textcolor[rgb]{0.10,0.09,0.49}{##1}}}
\expandafter\def\csname PY@tok@il\endcsname{\def\PY@tc##1{\textcolor[rgb]{0.40,0.40,0.40}{##1}}}
\expandafter\def\csname PY@tok@go\endcsname{\def\PY@tc##1{\textcolor[rgb]{0.53,0.53,0.53}{##1}}}
\expandafter\def\csname PY@tok@cp\endcsname{\def\PY@tc##1{\textcolor[rgb]{0.74,0.48,0.00}{##1}}}
\expandafter\def\csname PY@tok@gi\endcsname{\def\PY@tc##1{\textcolor[rgb]{0.00,0.63,0.00}{##1}}}
\expandafter\def\csname PY@tok@gh\endcsname{\let\PY@bf=\textbf\def\PY@tc##1{\textcolor[rgb]{0.00,0.00,0.50}{##1}}}
\expandafter\def\csname PY@tok@ni\endcsname{\let\PY@bf=\textbf\def\PY@tc##1{\textcolor[rgb]{0.60,0.60,0.60}{##1}}}
\expandafter\def\csname PY@tok@nl\endcsname{\def\PY@tc##1{\textcolor[rgb]{0.63,0.63,0.00}{##1}}}
\expandafter\def\csname PY@tok@nn\endcsname{\let\PY@bf=\textbf\def\PY@tc##1{\textcolor[rgb]{0.00,0.00,1.00}{##1}}}
\expandafter\def\csname PY@tok@no\endcsname{\def\PY@tc##1{\textcolor[rgb]{0.53,0.00,0.00}{##1}}}
\expandafter\def\csname PY@tok@na\endcsname{\def\PY@tc##1{\textcolor[rgb]{0.49,0.56,0.16}{##1}}}
\expandafter\def\csname PY@tok@nb\endcsname{\def\PY@tc##1{\textcolor[rgb]{0.00,0.50,0.00}{##1}}}
\expandafter\def\csname PY@tok@nc\endcsname{\let\PY@bf=\textbf\def\PY@tc##1{\textcolor[rgb]{0.00,0.00,1.00}{##1}}}
\expandafter\def\csname PY@tok@nd\endcsname{\def\PY@tc##1{\textcolor[rgb]{0.67,0.13,1.00}{##1}}}
\expandafter\def\csname PY@tok@ne\endcsname{\let\PY@bf=\textbf\def\PY@tc##1{\textcolor[rgb]{0.82,0.25,0.23}{##1}}}
\expandafter\def\csname PY@tok@nf\endcsname{\def\PY@tc##1{\textcolor[rgb]{0.00,0.00,1.00}{##1}}}
\expandafter\def\csname PY@tok@si\endcsname{\let\PY@bf=\textbf\def\PY@tc##1{\textcolor[rgb]{0.73,0.40,0.53}{##1}}}
\expandafter\def\csname PY@tok@s2\endcsname{\def\PY@tc##1{\textcolor[rgb]{0.73,0.13,0.13}{##1}}}
\expandafter\def\csname PY@tok@nt\endcsname{\let\PY@bf=\textbf\def\PY@tc##1{\textcolor[rgb]{0.00,0.50,0.00}{##1}}}
\expandafter\def\csname PY@tok@nv\endcsname{\def\PY@tc##1{\textcolor[rgb]{0.10,0.09,0.49}{##1}}}
\expandafter\def\csname PY@tok@s1\endcsname{\def\PY@tc##1{\textcolor[rgb]{0.73,0.13,0.13}{##1}}}
\expandafter\def\csname PY@tok@dl\endcsname{\def\PY@tc##1{\textcolor[rgb]{0.73,0.13,0.13}{##1}}}
\expandafter\def\csname PY@tok@ch\endcsname{\let\PY@it=\textit\def\PY@tc##1{\textcolor[rgb]{0.25,0.50,0.50}{##1}}}
\expandafter\def\csname PY@tok@m\endcsname{\def\PY@tc##1{\textcolor[rgb]{0.40,0.40,0.40}{##1}}}
\expandafter\def\csname PY@tok@gp\endcsname{\let\PY@bf=\textbf\def\PY@tc##1{\textcolor[rgb]{0.00,0.00,0.50}{##1}}}
\expandafter\def\csname PY@tok@sh\endcsname{\def\PY@tc##1{\textcolor[rgb]{0.73,0.13,0.13}{##1}}}
\expandafter\def\csname PY@tok@ow\endcsname{\let\PY@bf=\textbf\def\PY@tc##1{\textcolor[rgb]{0.67,0.13,1.00}{##1}}}
\expandafter\def\csname PY@tok@sx\endcsname{\def\PY@tc##1{\textcolor[rgb]{0.00,0.50,0.00}{##1}}}
\expandafter\def\csname PY@tok@bp\endcsname{\def\PY@tc##1{\textcolor[rgb]{0.00,0.50,0.00}{##1}}}
\expandafter\def\csname PY@tok@c1\endcsname{\let\PY@it=\textit\def\PY@tc##1{\textcolor[rgb]{0.25,0.50,0.50}{##1}}}
\expandafter\def\csname PY@tok@fm\endcsname{\def\PY@tc##1{\textcolor[rgb]{0.00,0.00,1.00}{##1}}}
\expandafter\def\csname PY@tok@o\endcsname{\def\PY@tc##1{\textcolor[rgb]{0.40,0.40,0.40}{##1}}}
\expandafter\def\csname PY@tok@kc\endcsname{\let\PY@bf=\textbf\def\PY@tc##1{\textcolor[rgb]{0.00,0.50,0.00}{##1}}}
\expandafter\def\csname PY@tok@c\endcsname{\let\PY@it=\textit\def\PY@tc##1{\textcolor[rgb]{0.25,0.50,0.50}{##1}}}
\expandafter\def\csname PY@tok@mf\endcsname{\def\PY@tc##1{\textcolor[rgb]{0.40,0.40,0.40}{##1}}}
\expandafter\def\csname PY@tok@err\endcsname{\def\PY@bc##1{\setlength{\fboxsep}{0pt}\fcolorbox[rgb]{1.00,0.00,0.00}{1,1,1}{\strut ##1}}}
\expandafter\def\csname PY@tok@mb\endcsname{\def\PY@tc##1{\textcolor[rgb]{0.40,0.40,0.40}{##1}}}
\expandafter\def\csname PY@tok@ss\endcsname{\def\PY@tc##1{\textcolor[rgb]{0.10,0.09,0.49}{##1}}}
\expandafter\def\csname PY@tok@sr\endcsname{\def\PY@tc##1{\textcolor[rgb]{0.73,0.40,0.53}{##1}}}
\expandafter\def\csname PY@tok@mo\endcsname{\def\PY@tc##1{\textcolor[rgb]{0.40,0.40,0.40}{##1}}}
\expandafter\def\csname PY@tok@kd\endcsname{\let\PY@bf=\textbf\def\PY@tc##1{\textcolor[rgb]{0.00,0.50,0.00}{##1}}}
\expandafter\def\csname PY@tok@mi\endcsname{\def\PY@tc##1{\textcolor[rgb]{0.40,0.40,0.40}{##1}}}
\expandafter\def\csname PY@tok@kn\endcsname{\let\PY@bf=\textbf\def\PY@tc##1{\textcolor[rgb]{0.00,0.50,0.00}{##1}}}
\expandafter\def\csname PY@tok@cpf\endcsname{\let\PY@it=\textit\def\PY@tc##1{\textcolor[rgb]{0.25,0.50,0.50}{##1}}}
\expandafter\def\csname PY@tok@kr\endcsname{\let\PY@bf=\textbf\def\PY@tc##1{\textcolor[rgb]{0.00,0.50,0.00}{##1}}}
\expandafter\def\csname PY@tok@s\endcsname{\def\PY@tc##1{\textcolor[rgb]{0.73,0.13,0.13}{##1}}}
\expandafter\def\csname PY@tok@kp\endcsname{\def\PY@tc##1{\textcolor[rgb]{0.00,0.50,0.00}{##1}}}
\expandafter\def\csname PY@tok@w\endcsname{\def\PY@tc##1{\textcolor[rgb]{0.73,0.73,0.73}{##1}}}
\expandafter\def\csname PY@tok@kt\endcsname{\def\PY@tc##1{\textcolor[rgb]{0.69,0.00,0.25}{##1}}}
\expandafter\def\csname PY@tok@sc\endcsname{\def\PY@tc##1{\textcolor[rgb]{0.73,0.13,0.13}{##1}}}
\expandafter\def\csname PY@tok@sb\endcsname{\def\PY@tc##1{\textcolor[rgb]{0.73,0.13,0.13}{##1}}}
\expandafter\def\csname PY@tok@sa\endcsname{\def\PY@tc##1{\textcolor[rgb]{0.73,0.13,0.13}{##1}}}
\expandafter\def\csname PY@tok@k\endcsname{\let\PY@bf=\textbf\def\PY@tc##1{\textcolor[rgb]{0.00,0.50,0.00}{##1}}}
\expandafter\def\csname PY@tok@se\endcsname{\let\PY@bf=\textbf\def\PY@tc##1{\textcolor[rgb]{0.73,0.40,0.13}{##1}}}
\expandafter\def\csname PY@tok@sd\endcsname{\let\PY@it=\textit\def\PY@tc##1{\textcolor[rgb]{0.73,0.13,0.13}{##1}}}

\def\PYZbs{\char`\\}
\def\PYZus{\char`\_}
\def\PYZob{\char`\{}
\def\PYZcb{\char`\}}
\def\PYZca{\char`\^}
\def\PYZam{\char`\&}
\def\PYZlt{\char`\<}
\def\PYZgt{\char`\>}
\def\PYZsh{\char`\#}
\def\PYZpc{\char`\%}
\def\PYZdl{\char`\$}
\def\PYZhy{\char`\-}
\def\PYZsq{\char`\'}
\def\PYZdq{\char`\"}
\def\PYZti{\char`\~}
% for compatibility with earlier versions
\def\PYZat{@}
\def\PYZlb{[}
\def\PYZrb{]}
\makeatother


    % Exact colors from NB
    \definecolor{incolor}{rgb}{0.0, 0.0, 0.5}
    \definecolor{outcolor}{rgb}{0.545, 0.0, 0.0}



    
    % Prevent overflowing lines due to hard-to-break entities
    \sloppy 
    % Setup hyperref package
    \hypersetup{
      breaklinks=true,  % so long urls are correctly broken across lines
      colorlinks=true,
      urlcolor=urlcolor,
      linkcolor=linkcolor,
      citecolor=citecolor,
      }
    % Slightly bigger margins than the latex defaults
    
    \geometry{verbose,tmargin=1in,bmargin=1in,lmargin=1in,rmargin=1in}
    
    

    \begin{document}
    
    
    \maketitle
    
    

    
    Table of Contents{}

{{1~~}Data types}

{{2~~}Data structures}

{{2.1~~}Primary data structures}

{{2.1.1~~}Vectors (atomic vectors)}

{{2.1.2~~}Lists (recursive vectors)}

{{2.2~~}Secondary data structures}

{{2.2.1~~}matrix and arrays}

{{2.2.1.1~~}using cbind and rbind}

{{2.2.1.2~~}Vector and Matrix Operations}

{{2.2.2~~}data frame}

{{2.2.2.1~~}using expand.grid()}

{{2.2.2.2~~}using cbind and rbind}

{{3~~}Special values}

{{3.1~~}NA}

{{3.2~~}NULL}

{{3.3~~}Inf}

{{3.4~~}NaN}

{{4~~}Coercion}

{{4.1~~}Internal coercion}

{{4.2~~}External coercion and testing objects}

{{5~~}Reshaping R Objects}

{{5.1~~}for vectors and matrices}

{{5.2~~}for dataframes}

    \begin{figure}[htbp]
\centering
\includegraphics{https://i.imgur.com/PJjcuPu.jpg}
\caption{Imgur}
\end{figure}

    ** Objects ** \textgreater{} Objects are simply a definition for a type
of data to be stored\\
e.g., data vector, matrix, array, data frame, list, function

    \subsection{Data types}\label{data-types}

(basic atomic classes)

    \begin{itemize}
\tightlist
\item
  Numeric (e.g, 9, 7.2, pi)
\item
  Integers (e.g,, 3L, as.integer(5))
\item
  Logical (e.g., TRUE, FALSE)
\item
  Complex (e.g, -3-87i, 1 + 0i, 1 + 4i)
\item
  Characters (e.g, "apple", "red")
\item
  Factor (e.g, Male, Female \textbf{OR} low, medium, high)
\item
  Dates and Times (e.g, "2008-01-05", "2018-06-16")
\end{itemize}

    \subsection{Data structures}\label{data-structures}

A perticular way of organizing data

    \subsubsection{Primary data structures}\label{primary-data-structures}

    Vectors can be of two types: * Atomic vectors (basic data types in R and
is pretty much the workhorse of R) * Recursive Vectors or lists (List is
a special vector. Each element can be a different class)

    \paragraph{Vectors (atomic vectors)}\label{vectors-atomic-vectors}

    \begin{itemize}
\tightlist
\item
  A vector can be a vector of characters, logical, integers or
  numeric.\\
  You can create vectors by concatenating them using the \texttt{c()}
  function.\\
  Various examples:\\
  \texttt{x\ \&lt;-\ c(1,\ 2,\ 3)}\\
  x is a numeric vector. These are the most common kind. They are
  numeric objects and are treated as double precision real numbers. To
  explicitly create integers, add a L at the end.\\
  \texttt{x1\ \&lt;-\ c(1L,\ 2L,\ 3L)} You can also have logical
  vectors.\\
  \texttt{y\ \&lt;-\ c(TRUE,\ TRUE,\ FALSE,\ FALSE)}\\
  Finally you can have character vectors:\\
  \texttt{z\ \&lt;-\ c("Alec",\ "Dan",\ "Rob",\ "Rich")}
\end{itemize}

    \begin{verbatim}
#The general pattern is vector(class of object, length). 
Create an empty vector with vector()
x &lt;- vector()
# with a pre-defined length
x &lt;- vector(length = 10)
# with a length and type
vector("character", length = 10)
vector("numeric", length = 10)
vector("integer", length = 10)
vector("logical", length = 10)
\end{verbatim}

    \textbf{Vectors may only have one type}\\
* atomic vectors are homogeneous. This means that every element within a
single atomic vector has to be of the same type * R will create a
resulting vector that is the least common denominator. The coercion will
move towards the one that's easiest to coerce to. Guess what the
following do without running them first

\begin{verbatim}
xx &lt;- c(1.7, "a")
xx &lt;- c(TRUE, 2)
xx &lt;- c("a", TRUE)
This is called implicit coercion.
\end{verbatim}

The coersion rule goes\\
\emph{logical -\textgreater{} integer -\textgreater{} numeric
-\textgreater{} complex -\textgreater{} character.}

    \textbf{Some Function for Vectors}

\begin{longtable}[c]{@{}ll@{}}
\toprule
\begin{minipage}[b]{0.13\columnwidth}\raggedright\strut
Functions
\strut\end{minipage} &
\begin{minipage}[b]{0.81\columnwidth}\raggedright\strut
Description
\strut\end{minipage}\tabularnewline
\midrule
\endhead
\begin{minipage}[t]{0.13\columnwidth}\raggedright\strut
\texttt{c()}
\strut\end{minipage} &
\begin{minipage}[t]{0.81\columnwidth}\raggedright\strut
combines values, vectors, and/or lists to create new objects
\strut\end{minipage}\tabularnewline
\begin{minipage}[t]{0.13\columnwidth}\raggedright\strut
\texttt{unique()}
\strut\end{minipage} &
\begin{minipage}[t]{0.81\columnwidth}\raggedright\strut
returns a vector containing one element for each unique value in the
vector
\strut\end{minipage}\tabularnewline
\begin{minipage}[t]{0.13\columnwidth}\raggedright\strut
\texttt{duplicated()}
\strut\end{minipage} &
\begin{minipage}[t]{0.81\columnwidth}\raggedright\strut
returns a logical vector which tells if elements of a vector are
duplicated with regard to previous one
\strut\end{minipage}\tabularnewline
\begin{minipage}[t]{0.13\columnwidth}\raggedright\strut
\texttt{rev()}
\strut\end{minipage} &
\begin{minipage}[t]{0.81\columnwidth}\raggedright\strut
reverse the order of element in a vector
\strut\end{minipage}\tabularnewline
\begin{minipage}[t]{0.13\columnwidth}\raggedright\strut
\texttt{sort()}
\strut\end{minipage} &
\begin{minipage}[t]{0.81\columnwidth}\raggedright\strut
sorts the elements in a vector
\strut\end{minipage}\tabularnewline
\begin{minipage}[t]{0.13\columnwidth}\raggedright\strut
\texttt{append()}
\strut\end{minipage} &
\begin{minipage}[t]{0.81\columnwidth}\raggedright\strut
append or insert elements in a vector.
\strut\end{minipage}\tabularnewline
\begin{minipage}[t]{0.13\columnwidth}\raggedright\strut
\texttt{sum()}
\strut\end{minipage} &
\begin{minipage}[t]{0.81\columnwidth}\raggedright\strut
sum of the elements of a vector
\strut\end{minipage}\tabularnewline
\begin{minipage}[t]{0.13\columnwidth}\raggedright\strut
\texttt{min()}
\strut\end{minipage} &
\begin{minipage}[t]{0.81\columnwidth}\raggedright\strut
minimum value in a vector
\strut\end{minipage}\tabularnewline
\begin{minipage}[t]{0.13\columnwidth}\raggedright\strut
\texttt{max()}
\strut\end{minipage} &
\begin{minipage}[t]{0.81\columnwidth}\raggedright\strut
maximum value in a vector
\strut\end{minipage}\tabularnewline
\begin{minipage}[t]{0.13\columnwidth}\raggedright\strut
\texttt{cumsum}
\strut\end{minipage} &
\begin{minipage}[t]{0.81\columnwidth}\raggedright\strut
cumulative sum
\strut\end{minipage}\tabularnewline
\begin{minipage}[t]{0.13\columnwidth}\raggedright\strut
\texttt{diff}
\strut\end{minipage} &
\begin{minipage}[t]{0.81\columnwidth}\raggedright\strut
x{[}i+1{]} - x{[}i{]}
\strut\end{minipage}\tabularnewline
\begin{minipage}[t]{0.13\columnwidth}\raggedright\strut
\texttt{prod}
\strut\end{minipage} &
\begin{minipage}[t]{0.81\columnwidth}\raggedright\strut
product
\strut\end{minipage}\tabularnewline
\begin{minipage}[t]{0.13\columnwidth}\raggedright\strut
\texttt{cumprod}
\strut\end{minipage} &
\begin{minipage}[t]{0.81\columnwidth}\raggedright\strut
cumulative product
\strut\end{minipage}\tabularnewline
\begin{minipage}[t]{0.13\columnwidth}\raggedright\strut
\texttt{mean}
\strut\end{minipage} &
\begin{minipage}[t]{0.81\columnwidth}\raggedright\strut
average
\strut\end{minipage}\tabularnewline
\begin{minipage}[t]{0.13\columnwidth}\raggedright\strut
\texttt{median}
\strut\end{minipage} &
\begin{minipage}[t]{0.81\columnwidth}\raggedright\strut
median
\strut\end{minipage}\tabularnewline
\begin{minipage}[t]{0.13\columnwidth}\raggedright\strut
\texttt{range}
\strut\end{minipage} &
\begin{minipage}[t]{0.81\columnwidth}\raggedright\strut
range (minimum and maximum)
\strut\end{minipage}\tabularnewline
\begin{minipage}[t]{0.13\columnwidth}\raggedright\strut
\texttt{order}
\strut\end{minipage} &
\begin{minipage}[t]{0.81\columnwidth}\raggedright\strut
order
\strut\end{minipage}\tabularnewline
\begin{minipage}[t]{0.13\columnwidth}\raggedright\strut
\texttt{rank}
\strut\end{minipage} &
\begin{minipage}[t]{0.81\columnwidth}\raggedright\strut
rank
\strut\end{minipage}\tabularnewline
\begin{minipage}[t]{0.13\columnwidth}\raggedright\strut
\texttt{sample}
\strut\end{minipage} &
\begin{minipage}[t]{0.81\columnwidth}\raggedright\strut
random sample
\strut\end{minipage}\tabularnewline
\begin{minipage}[t]{0.13\columnwidth}\raggedright\strut
\texttt{quartile}
\strut\end{minipage} &
\begin{minipage}[t]{0.81\columnwidth}\raggedright\strut
percentile
\strut\end{minipage}\tabularnewline
\begin{minipage}[t]{0.13\columnwidth}\raggedright\strut
\texttt{var}
\strut\end{minipage} &
\begin{minipage}[t]{0.81\columnwidth}\raggedright\strut
variance, covariance
\strut\end{minipage}\tabularnewline
\begin{minipage}[t]{0.13\columnwidth}\raggedright\strut
\texttt{sd}
\strut\end{minipage} &
\begin{minipage}[t]{0.81\columnwidth}\raggedright\strut
standard deviation
\strut\end{minipage}\tabularnewline
\bottomrule
\end{longtable}

    \begin{verbatim}
(x &lt;- c(sort(sample(1:20, 9)), NA))
(y &lt;- c(sort(sample(3:23, 7)), NA))
union(x, y)
intersect(x, y)
setdiff(x, y)
setdiff(y, x)
setequal(x, y)
which.min(x)
which.max(x)
match(x,y)
\end{verbatim}

    \paragraph{Lists (recursive vectors)}\label{lists-recursive-vectors}

    \begin{itemize}
\tightlist
\item
  List is a special vector. Each element can be a different class.
\item
  lists act as containers
\item
  Unlike atomic vectors, its contents are not restricted to a single
  type
\item
  a list can be anything, and two elements within a list can be of
  different types!
\item
  Lists are sometimes called recursive vectors, because a list can
  contain other lists
\end{itemize}

    ** Create a ists using list function**\\
\texttt{x\ \&lt;-\ list(1,\ "a",\ TRUE,\ 1+4i)}

    \subsubsection{Secondary data
structures}\label{secondary-data-structures}

    R also has many data structures. These include * matrix and arrays *
data frame

    \begin{figure}[htbp]
\centering
\includegraphics{https://i.imgur.com/c73sZAd.png?1}
\caption{Imgur}
\end{figure}

    \paragraph{matrix and arrays}\label{matrix-and-arrays}

    \begin{itemize}
\tightlist
\item
  Stacking multiple matrices
\item
  Matrices in R can be thought of as vectors indexed using two indices
  instead of one
\end{itemize}

    \begin{Verbatim}[commandchars=\\\{\}]
{\color{incolor}In [{\color{incolor}1}]:} m \PY{o}{\PYZlt{}\PYZhy{}} \PY{k+kt}{matrix}\PY{p}{(}\PY{k+kt}{c}\PY{p}{(}\PY{l+m}{47}\PY{p}{,}\PY{l+m}{48}\PY{p}{,}\PY{l+m}{52}\PY{p}{,}\PY{l+m}{36}\PY{p}{,}\PY{l+m}{89}\PY{p}{,}\PY{l+m}{45}\PY{p}{,}\PY{l+m}{1}\PY{p}{,}\PY{l+m}{2}\PY{p}{,}\PY{l+m}{\PYZhy{}25}\PY{p}{)}\PY{p}{,} nrow \PY{o}{=} \PY{l+m}{3}\PY{p}{,} ncol \PY{o}{=} \PY{l+m}{3}\PY{p}{)}
        m
        \PY{c+c1}{\PYZsh{} this code creates a 3 by 3 matrix}
\end{Verbatim}


    \begin{tabular}{lll}
	 47  & 36  &   1\\
	 48  & 89  &   2\\
	 52  & 45  & -25\\
\end{tabular}


    
    \subparagraph{using cbind and rbind}\label{using-cbind-and-rbind}

(to create a matrix)

    \begin{Verbatim}[commandchars=\\\{\}]
{\color{incolor}In [{\color{incolor}44}]:} \PY{c+c1}{\PYZsh{} 3 column matrix with column dimnames}
         m \PY{o}{\PYZlt{}\PYZhy{}} \PY{k+kp}{cbind}\PY{p}{(}Index \PY{o}{=} \PY{k+kt}{c}\PY{p}{(}\PY{l+m}{1}\PY{o}{:}\PY{l+m}{3}\PY{p}{)}\PY{p}{,} Age \PY{o}{=} \PY{k+kt}{c}\PY{p}{(}\PY{l+m}{30}\PY{p}{,} \PY{l+m}{45}\PY{p}{,} \PY{l+m}{34}\PY{p}{)}\PY{p}{,} Salary \PY{o}{=} \PY{k+kt}{c}\PY{p}{(}\PY{l+m}{500}\PY{p}{,} \PY{l+m}{600}\PY{p}{,} \PY{l+m}{550}\PY{p}{)}\PY{p}{)} 
         \PY{k+kp}{class}\PY{p}{(}m\PY{p}{)}
         m
\end{Verbatim}


    'matrix'

    
    \begin{tabular}{lll}
 Index & Age & Salary\\
\hline
	 1   & 30  & 500\\
	 2   & 45  & 600\\
	 3   & 34  & 550\\
\end{tabular}


    
    \begin{itemize}
\tightlist
\item
  If a higher dimension vector is desired, then use the array() function
  to generate the n-dimensional object. A 3x3x3 array can be created as
  follows:
\end{itemize}

    \begin{Verbatim}[commandchars=\\\{\}]
{\color{incolor}In [{\color{incolor}45}]:} m \PY{o}{\PYZlt{}\PYZhy{}} \PY{k+kt}{array}\PY{p}{(}\PY{l+m}{1}\PY{o}{:}\PY{l+m}{27}\PY{p}{,} dim\PY{o}{=}\PY{k+kt}{c}\PY{p}{(}\PY{l+m}{3}\PY{p}{,}\PY{l+m}{3}\PY{p}{,}\PY{l+m}{3}\PY{p}{)}\PY{p}{)}
         \PY{k+kp}{print}\PY{p}{(}m\PY{p}{)}
\end{Verbatim}


    \begin{Verbatim}[commandchars=\\\{\}]
, , 1

     [,1] [,2] [,3]
[1,]    1    4    7
[2,]    2    5    8
[3,]    3    6    9

, , 2

     [,1] [,2] [,3]
[1,]   10   13   16
[2,]   11   14   17
[3,]   12   15   18

, , 3

     [,1] [,2] [,3]
[1,]   19   22   25
[2,]   20   23   26
[3,]   21   24   27


    \end{Verbatim}

    \subparagraph{Vector and Matrix
Operations}\label{vector-and-matrix-operations}

    \begin{figure}[htbp]
\centering
\includegraphics{https://i.imgur.com/HmnOp1F.png}
\caption{Imgur}
\end{figure}

    \begin{figure}[htbp]
\centering
\includegraphics{https://i.imgur.com/xe79KhH.png}
\caption{Imgur}
\end{figure}

    \paragraph{data frame}\label{data-frame}

    \begin{itemize}
\tightlist
\item
  unlike a matrix, a data frame can mix data types across columns
\end{itemize}

    \begin{Verbatim}[commandchars=\\\{\}]
{\color{incolor}In [{\color{incolor}46}]:} Name \PY{o}{\PYZlt{}\PYZhy{}} \PY{k+kt}{c}\PY{p}{(}\PY{l+s}{\PYZdq{}}\PY{l+s}{a1\PYZdq{}}\PY{p}{,} \PY{l+s}{\PYZdq{}}\PY{l+s}{a2\PYZdq{}}\PY{p}{,} \PY{l+s}{\PYZdq{}}\PY{l+s}{b3\PYZdq{}}\PY{p}{)}
         Value1 \PY{o}{\PYZlt{}\PYZhy{}} \PY{k+kt}{c}\PY{p}{(}\PY{l+m}{23}\PY{p}{,} \PY{l+m}{4}\PY{p}{,} \PY{l+m}{12}\PY{p}{)}
         Value2 \PY{o}{\PYZlt{}\PYZhy{}} \PY{k+kt}{c}\PY{p}{(}\PY{l+m}{1}\PY{p}{,}\PY{l+m}{45}\PY{p}{,}\PY{l+m}{5}\PY{p}{)}
         dat \PY{o}{\PYZlt{}\PYZhy{}} \PY{k+kt}{data.frame}\PY{p}{(}Name\PY{p}{,} Value1\PY{p}{,} Value2\PY{p}{)}
         dat
\end{Verbatim}


    \begin{tabular}{r|lll}
 Name & Value1 & Value2\\
\hline
	 a1 & 23 &  1\\
	 a2 &  4 & 45\\
	 b3 & 12 &  5\\
\end{tabular}


    
    \subparagraph{using expand.grid()}\label{using-expand.grid}

(to create a data frame) * The function \texttt{expand.grid} gives us
all the combinations of entries of two vectors.

    For example:\\
all combinations of \texttt{blue\ and\ black\ pants} and
\texttt{white,\ grey\ and\ plaid\ shirts}

    \begin{Verbatim}[commandchars=\\\{\}]
{\color{incolor}In [{\color{incolor}47}]:} eg \PY{o}{\PYZlt{}\PYZhy{}} \PY{k+kp}{expand.grid}\PY{p}{(}pants \PY{o}{=} \PY{k+kt}{c}\PY{p}{(}\PY{l+s}{\PYZdq{}}\PY{l+s}{blue\PYZdq{}}\PY{p}{,} \PY{l+s}{\PYZdq{}}\PY{l+s}{black\PYZdq{}}\PY{p}{)}\PY{p}{,} shirt \PY{o}{=} \PY{k+kt}{c}\PY{p}{(}\PY{l+s}{\PYZdq{}}\PY{l+s}{white\PYZdq{}}\PY{p}{,} \PY{l+s}{\PYZdq{}}\PY{l+s}{grey\PYZdq{}}\PY{p}{,} \PY{l+s}{\PYZdq{}}\PY{l+s}{plaid\PYZdq{}}\PY{p}{)}\PY{p}{)}
         \PY{k+kp}{class}\PY{p}{(}eg\PY{p}{)}
\end{Verbatim}


    'data.frame'

    
    \begin{Verbatim}[commandchars=\\\{\}]
{\color{incolor}In [{\color{incolor}48}]:} \PY{c+c1}{\PYZsh{}generate a deck of cards}
         suits \PY{o}{\PYZlt{}\PYZhy{}} \PY{k+kt}{c}\PY{p}{(}\PY{l+s}{\PYZdq{}}\PY{l+s}{Diamonds\PYZdq{}}\PY{p}{,} \PY{l+s}{\PYZdq{}}\PY{l+s}{Clubs\PYZdq{}}\PY{p}{,} \PY{l+s}{\PYZdq{}}\PY{l+s}{Hearts\PYZdq{}}\PY{p}{,} \PY{l+s}{\PYZdq{}}\PY{l+s}{Spades\PYZdq{}}\PY{p}{)}
         numbers \PY{o}{\PYZlt{}\PYZhy{}} \PY{k+kt}{c}\PY{p}{(}\PY{l+s}{\PYZdq{}}\PY{l+s}{Ace\PYZdq{}}\PY{p}{,} \PY{l+s}{\PYZdq{}}\PY{l+s}{Deuce\PYZdq{}}\PY{p}{,} \PY{l+s}{\PYZdq{}}\PY{l+s}{Three\PYZdq{}}\PY{p}{,} \PY{l+s}{\PYZdq{}}\PY{l+s}{Four\PYZdq{}}\PY{p}{,} \PY{l+s}{\PYZdq{}}\PY{l+s}{Five\PYZdq{}}\PY{p}{,} \PY{l+s}{\PYZdq{}}\PY{l+s}{Six\PYZdq{}}\PY{p}{,} \PY{l+s}{\PYZdq{}}\PY{l+s}{Seven\PYZdq{}}\PY{p}{,} \PY{l+s}{\PYZdq{}}\PY{l+s}{Eight\PYZdq{}}\PY{p}{,} \PY{l+s}{\PYZdq{}}\PY{l+s}{Nine\PYZdq{}}\PY{p}{,} \PY{l+s}{\PYZdq{}}\PY{l+s}{Ten\PYZdq{}}\PY{p}{,} \PY{l+s}{\PYZdq{}}\PY{l+s}{Jack\PYZdq{}}\PY{p}{,} \PY{l+s}{\PYZdq{}}\PY{l+s}{Queen\PYZdq{}}\PY{p}{,} \PY{l+s}{\PYZdq{}}\PY{l+s}{King\PYZdq{}}\PY{p}{)}
         deck \PY{o}{\PYZlt{}\PYZhy{}} \PY{k+kp}{expand.grid}\PY{p}{(}number\PY{o}{=}numbers\PY{p}{,} suit\PY{o}{=}suits\PY{p}{)}
         \PY{k+kp}{class}\PY{p}{(}deck\PY{p}{)}
         \PY{k+kp}{print}\PY{p}{(}deck\PY{p}{)}
         deck \PY{o}{\PYZlt{}\PYZhy{}} \PY{k+kp}{paste}\PY{p}{(}deck\PY{o}{\PYZdl{}}number\PY{p}{,} deck\PY{o}{\PYZdl{}}suit\PY{p}{)}
         \PY{k+kp}{class}\PY{p}{(}deck\PY{p}{)}
         \PY{k+kp}{print}\PY{p}{(}deck\PY{p}{)}
\end{Verbatim}


    'data.frame'

    
    \begin{Verbatim}[commandchars=\\\{\}]
   number     suit
1     Ace Diamonds
2   Deuce Diamonds
3   Three Diamonds
4    Four Diamonds
5    Five Diamonds
6     Six Diamonds
7   Seven Diamonds
8   Eight Diamonds
9    Nine Diamonds
10    Ten Diamonds
11   Jack Diamonds
12  Queen Diamonds
13   King Diamonds
14    Ace    Clubs
15  Deuce    Clubs
16  Three    Clubs
17   Four    Clubs
18   Five    Clubs
19    Six    Clubs
20  Seven    Clubs
21  Eight    Clubs
22   Nine    Clubs
23    Ten    Clubs
24   Jack    Clubs
25  Queen    Clubs
26   King    Clubs
27    Ace   Hearts
28  Deuce   Hearts
29  Three   Hearts
30   Four   Hearts
31   Five   Hearts
32    Six   Hearts
33  Seven   Hearts
34  Eight   Hearts
35   Nine   Hearts
36    Ten   Hearts
37   Jack   Hearts
38  Queen   Hearts
39   King   Hearts
40    Ace   Spades
41  Deuce   Spades
42  Three   Spades
43   Four   Spades
44   Five   Spades
45    Six   Spades
46  Seven   Spades
47  Eight   Spades
48   Nine   Spades
49    Ten   Spades
50   Jack   Spades
51  Queen   Spades
52   King   Spades

    \end{Verbatim}

    'character'

    
    \begin{Verbatim}[commandchars=\\\{\}]
 [1] "Ace Diamonds"   "Deuce Diamonds" "Three Diamonds" "Four Diamonds" 
 [5] "Five Diamonds"  "Six Diamonds"   "Seven Diamonds" "Eight Diamonds"
 [9] "Nine Diamonds"  "Ten Diamonds"   "Jack Diamonds"  "Queen Diamonds"
[13] "King Diamonds"  "Ace Clubs"      "Deuce Clubs"    "Three Clubs"   
[17] "Four Clubs"     "Five Clubs"     "Six Clubs"      "Seven Clubs"   
[21] "Eight Clubs"    "Nine Clubs"     "Ten Clubs"      "Jack Clubs"    
[25] "Queen Clubs"    "King Clubs"     "Ace Hearts"     "Deuce Hearts"  
[29] "Three Hearts"   "Four Hearts"    "Five Hearts"    "Six Hearts"    
[33] "Seven Hearts"   "Eight Hearts"   "Nine Hearts"    "Ten Hearts"    
[37] "Jack Hearts"    "Queen Hearts"   "King Hearts"    "Ace Spades"    
[41] "Deuce Spades"   "Three Spades"   "Four Spades"    "Five Spades"   
[45] "Six Spades"     "Seven Spades"   "Eight Spades"   "Nine Spades"   
[49] "Ten Spades"     "Jack Spades"    "Queen Spades"   "King Spades"   

    \end{Verbatim}

    \subparagraph{using cbind and rbind}\label{using-cbind-and-rbind}

(to use for a data frame)

    \begin{Verbatim}[commandchars=\\\{\}]
{\color{incolor}In [{\color{incolor}27}]:} df \PY{o}{\PYZlt{}\PYZhy{}} \PY{k+kt}{data.frame}\PY{p}{(}x \PY{o}{=} \PY{l+m}{1}\PY{o}{:}\PY{l+m}{3}\PY{p}{,} y \PY{o}{=} \PY{k+kt}{c}\PY{p}{(}\PY{l+s}{\PYZdq{}}\PY{l+s}{a\PYZdq{}}\PY{p}{,} \PY{l+s}{\PYZdq{}}\PY{l+s}{b\PYZdq{}}\PY{p}{,} \PY{l+s}{\PYZdq{}}\PY{l+s}{c\PYZdq{}}\PY{p}{)}\PY{p}{)}
         df
         df1 \PY{o}{\PYZlt{}\PYZhy{}} \PY{k+kp}{cbind}\PY{p}{(}\PY{l+m}{1}\PY{p}{,}  df\PY{p}{)} 
         \PY{k+kp}{class}\PY{p}{(}df1\PY{p}{)}
\end{Verbatim}


    \begin{tabular}{r|ll}
 x & y\\
\hline
	 1 & a\\
	 2 & b\\
	 3 & c\\
\end{tabular}


    
    'data.frame'

    
    \begin{Verbatim}[commandchars=\\\{\}]
{\color{incolor}In [{\color{incolor}50}]:} df \PY{o}{\PYZlt{}\PYZhy{}} \PY{k+kt}{data.frame}\PY{p}{(}a \PY{o}{=} \PY{k+kt}{c}\PY{p}{(}\PY{l+m}{1}\PY{o}{:}\PY{l+m}{5}\PY{p}{)}\PY{p}{,} b \PY{o}{=} \PY{p}{(}\PY{l+m}{1}\PY{o}{:}\PY{l+m}{5}\PY{p}{)}\PY{o}{\PYZca{}}\PY{l+m}{2}\PY{p}{)}
         df
         \PY{k+kp}{rbind}\PY{p}{(}df\PY{p}{,} \PY{k+kt}{c}\PY{p}{(}\PY{l+m}{2}\PY{p}{,} \PY{l+m}{3}\PY{p}{)}\PY{p}{,} \PY{k+kt}{c}\PY{p}{(}\PY{l+m}{5}\PY{p}{,} \PY{l+m}{6}\PY{p}{)}\PY{p}{)}
\end{Verbatim}


    \begin{tabular}{r|ll}
 a & b\\
\hline
	 1  &  1\\
	 2  &  4\\
	 3  &  9\\
	 4  & 16\\
	 5  & 25\\
\end{tabular}


    
    \begin{tabular}{r|ll}
 a & b\\
\hline
	 1  &  1\\
	 2  &  4\\
	 3  &  9\\
	 4  & 16\\
	 5  & 25\\
	 2  &  3\\
	 5  &  6\\
\end{tabular}


    
    \subsection{Special values}\label{special-values}

\texttt{NA,\ NULL,\ ±Inf\ and\ NaN}

    \subsubsection{NA}\label{na}

\begin{itemize}
\tightlist
\item
  NA Stands for not available
\item
  NA is a placeholder for a missing value
\end{itemize}

    \begin{Verbatim}[commandchars=\\\{\}]
{\color{incolor}In [{\color{incolor}1}]:} \PY{k+kc}{NA} \PY{o}{+} \PY{l+m}{2}
        \PY{k+kp}{sum}\PY{p}{(}\PY{k+kt}{c}\PY{p}{(}\PY{k+kc}{NA}\PY{p}{,} \PY{l+m}{4}\PY{p}{,} \PY{l+m}{6}\PY{p}{)}\PY{p}{)}
        median\PY{p}{(}\PY{k+kt}{c}\PY{p}{(}\PY{k+kc}{NA}\PY{p}{,} \PY{l+m}{4}\PY{p}{,} \PY{l+m}{8}\PY{p}{,} \PY{l+m}{4}\PY{p}{)}\PY{p}{,} na.rm \PY{o}{=} \PY{k+kc}{TRUE}\PY{p}{)}
        \PY{k+kp}{length}\PY{p}{(}\PY{k+kt}{c}\PY{p}{(}\PY{k+kc}{NA}\PY{p}{,} \PY{l+m}{2}\PY{p}{,} \PY{l+m}{3}\PY{p}{,} \PY{l+m}{4}\PY{p}{)}\PY{p}{)}
        \PY{l+m}{5} \PY{o}{==} \PY{k+kc}{NA}
        \PY{k+kc}{NA} \PY{o}{==} \PY{k+kc}{NA}
        \PY{k+kc}{TRUE} \PY{o}{|} \PY{k+kc}{NA}
\end{Verbatim}


    <NA>

    
    <NA>

    
    4

    
    4

    
    <NA>

    
    <NA>

    
    TRUE

    
    \subsubsection{NULL}\label{null}

\begin{itemize}
\tightlist
\item
  The class of NULL is null and has length 0
\item
  Does not take up any space in a vector
\item
  The function \texttt{is.null()} can be used to detect NULL variables.
\end{itemize}

    \begin{Verbatim}[commandchars=\\\{\}]
{\color{incolor}In [{\color{incolor}4}]:} \PY{k+kp}{length}\PY{p}{(}\PY{k+kt}{c}\PY{p}{(}\PY{l+m}{3}\PY{p}{,} \PY{l+m}{4}\PY{p}{,} \PY{k+kc}{NULL}\PY{p}{,} \PY{l+m}{1}\PY{p}{)}\PY{p}{)}
        \PY{k+kp}{sum}\PY{p}{(}\PY{k+kt}{c}\PY{p}{(}\PY{l+m}{5}\PY{p}{,} \PY{l+m}{1}\PY{p}{,} \PY{k+kc}{NULL}\PY{p}{,} \PY{l+m}{4}\PY{p}{)}\PY{p}{)}
        
        x \PY{o}{\PYZlt{}\PYZhy{}} \PY{k+kc}{NULL}
        \PY{k+kt}{c}\PY{p}{(}x\PY{p}{,} \PY{l+m}{5}\PY{p}{)}
\end{Verbatim}


    3

    
    10

    
    5

    
    \subsubsection{Inf}\label{inf}

\begin{itemize}
\tightlist
\item
  Inf is a valid \texttt{numeric} that results from calculations like
  division of a number by zero.
\item
  Since Inf is a numeric, operations between Inf and a finite numeric
  are well-defined and comparison operators work as expected.
\end{itemize}

    \begin{Verbatim}[commandchars=\\\{\}]
{\color{incolor}In [{\color{incolor}6}]:} \PY{l+m}{32}\PY{o}{/}\PY{l+m}{0}
        \PY{l+m}{5} \PY{o}{*} \PY{k+kc}{Inf}
        \PY{k+kc}{Inf} \PY{o}{\PYZhy{}} \PY{l+m}{2e+10}
        \PY{k+kc}{Inf} \PY{o}{+} \PY{k+kc}{Inf}
        \PY{l+m}{8} \PY{o}{\PYZlt{}} \PY{o}{\PYZhy{}}\PY{k+kc}{Inf}
        \PY{k+kc}{Inf} \PY{o}{==} \PY{k+kc}{Inf}
\end{Verbatim}


    Inf

    
    Inf

    
    Inf

    
    Inf

    
    FALSE

    
    TRUE

    
    \subsubsection{NaN}\label{nan}

\begin{itemize}
\tightlist
\item
  Stands for not a number.
\item
  unknown resulsts, but it is surely not a number
\item
  e.g like \texttt{0/0,\ Inf-Inf} and \texttt{Inf/Inf} result in NaN
\item
  Computations involving numbers and NaN always result in NaN
\end{itemize}

    \begin{Verbatim}[commandchars=\\\{\}]
{\color{incolor}In [{\color{incolor}5}]:} \PY{k+kc}{NaN} \PY{o}{+} \PY{l+m}{1}
        \PY{k+kp}{exp}\PY{p}{(}\PY{k+kc}{NaN}\PY{p}{)}
\end{Verbatim}


    NaN

    
    NaN

    
    \subsection{Coercion}\label{coercion}

    \subsubsection{Internal coercion}\label{internal-coercion}

    \begin{figure}[htbp]
\centering
\includegraphics{https://i.imgur.com/6uz95Br.jpg}
\caption{Imgur}
\end{figure}

    \begin{itemize}
\tightlist
\item
  All the data in an atomic vector must be of the same mode
\item
  If data are added so that modes are mixed, then the whole vector gets
  changed so that everything is of the most general mode
\end{itemize}

    \textbf{Example}

    \begin{Verbatim}[commandchars=\\\{\}]
{\color{incolor}In [{\color{incolor}17}]:} vec \PY{o}{\PYZlt{}\PYZhy{}} \PY{k+kt}{c}\PY{p}{(}\PY{l+m}{5}\PY{p}{,}\PY{l+m}{7}\PY{p}{,}\PY{l+m}{8}\PY{p}{,}\PY{l+m}{9}\PY{p}{,}\PY{l+m}{45}\PY{p}{,}\PY{l+m}{\PYZhy{}87}\PY{p}{)}
         \PY{k+kp}{print}\PY{p}{(}vec\PY{p}{)}
         \PY{k+kp}{mode}\PY{p}{(}vec\PY{p}{)}
         vec \PY{o}{\PYZlt{}\PYZhy{}} \PY{k+kp}{append}\PY{p}{(}vec\PY{p}{,} \PY{l+s}{\PYZdq{}}\PY{l+s}{hello\PYZdq{}}\PY{p}{)}
         \PY{k+kp}{print}\PY{p}{(}vec\PY{p}{)}
         \PY{k+kp}{mode}\PY{p}{(}vec\PY{p}{)}
\end{Verbatim}


    \begin{Verbatim}[commandchars=\\\{\}]
[1]   5   7   8   9  45 -87

    \end{Verbatim}

    'numeric'

    
    \begin{Verbatim}[commandchars=\\\{\}]
[1] "5"     "7"     "8"     "9"     "45"    "-87"   "hello"

    \end{Verbatim}

    'character'

    
    \subsubsection{External coercion and testing
objects}\label{external-coercion-and-testing-objects}

    \begin{figure}[htbp]
\centering
\includegraphics{https://i.imgur.com/hYYdi9O.png}
\caption{Imgur}
\end{figure}

    \subsection{Reshaping R Objects}\label{reshaping-r-objects}

    \subsubsection{for vectors and matrices}\label{for-vectors-and-matrices}

    \begin{Verbatim}[commandchars=\\\{\}]
{\color{incolor}In [{\color{incolor}6}]:} vec \PY{o}{\PYZlt{}\PYZhy{}} \PY{l+m}{1}\PY{o}{:}\PY{l+m}{12} \PY{c+c1}{\PYZsh{} a vector}
        \PY{k+kp}{print}\PY{p}{(}vec\PY{p}{)}
\end{Verbatim}


    \begin{Verbatim}[commandchars=\\\{\}]
 [1]  1  2  3  4  5  6  7  8  9 10 11 12

    \end{Verbatim}

    \begin{Verbatim}[commandchars=\\\{\}]
{\color{incolor}In [{\color{incolor}7}]:} mat \PY{o}{\PYZlt{}\PYZhy{}} \PY{k+kt}{matrix}\PY{p}{(} vec\PY{p}{,} nrow\PY{o}{=}\PY{l+m}{2}\PY{p}{)} \PY{c+c1}{\PYZsh{} a matrix}
        \PY{k+kp}{print}\PY{p}{(}mat\PY{p}{)}
\end{Verbatim}


    \begin{Verbatim}[commandchars=\\\{\}]
     [,1] [,2] [,3] [,4] [,5] [,6]
[1,]    1    3    5    7    9   11
[2,]    2    4    6    8   10   12

    \end{Verbatim}

    \begin{Verbatim}[commandchars=\\\{\}]
{\color{incolor}In [{\color{incolor}8}]:} \PY{k+kp}{dim}\PY{p}{(}mat\PY{p}{)} \PY{o}{\PYZlt{}\PYZhy{}} \PY{k+kc}{NULL}
        \PY{k+kp}{print}\PY{p}{(}mat\PY{p}{)} \PY{c+c1}{\PYZsh{} back to vector}
\end{Verbatim}


    \begin{Verbatim}[commandchars=\\\{\}]
 [1]  1  2  3  4  5  6  7  8  9 10 11 12

    \end{Verbatim}

    \subsubsection{for dataframes}\label{for-dataframes}

    \begin{Verbatim}[commandchars=\\\{\}]
{\color{incolor}In [{\color{incolor}9}]:} \PY{k+kp}{print}\PY{p}{(}mtcars\PY{p}{)}
\end{Verbatim}


    \begin{Verbatim}[commandchars=\\\{\}]
                     mpg cyl  disp  hp drat    wt  qsec vs am gear carb
Mazda RX4           21.0   6 160.0 110 3.90 2.620 16.46  0  1    4    4
Mazda RX4 Wag       21.0   6 160.0 110 3.90 2.875 17.02  0  1    4    4
Datsun 710          22.8   4 108.0  93 3.85 2.320 18.61  1  1    4    1
Hornet 4 Drive      21.4   6 258.0 110 3.08 3.215 19.44  1  0    3    1
Hornet Sportabout   18.7   8 360.0 175 3.15 3.440 17.02  0  0    3    2
Valiant             18.1   6 225.0 105 2.76 3.460 20.22  1  0    3    1
Duster 360          14.3   8 360.0 245 3.21 3.570 15.84  0  0    3    4
Merc 240D           24.4   4 146.7  62 3.69 3.190 20.00  1  0    4    2
Merc 230            22.8   4 140.8  95 3.92 3.150 22.90  1  0    4    2
Merc 280            19.2   6 167.6 123 3.92 3.440 18.30  1  0    4    4
Merc 280C           17.8   6 167.6 123 3.92 3.440 18.90  1  0    4    4
Merc 450SE          16.4   8 275.8 180 3.07 4.070 17.40  0  0    3    3
Merc 450SL          17.3   8 275.8 180 3.07 3.730 17.60  0  0    3    3
Merc 450SLC         15.2   8 275.8 180 3.07 3.780 18.00  0  0    3    3
Cadillac Fleetwood  10.4   8 472.0 205 2.93 5.250 17.98  0  0    3    4
Lincoln Continental 10.4   8 460.0 215 3.00 5.424 17.82  0  0    3    4
Chrysler Imperial   14.7   8 440.0 230 3.23 5.345 17.42  0  0    3    4
Fiat 128            32.4   4  78.7  66 4.08 2.200 19.47  1  1    4    1
Honda Civic         30.4   4  75.7  52 4.93 1.615 18.52  1  1    4    2
Toyota Corolla      33.9   4  71.1  65 4.22 1.835 19.90  1  1    4    1
Toyota Corona       21.5   4 120.1  97 3.70 2.465 20.01  1  0    3    1
Dodge Challenger    15.5   8 318.0 150 2.76 3.520 16.87  0  0    3    2
AMC Javelin         15.2   8 304.0 150 3.15 3.435 17.30  0  0    3    2
Camaro Z28          13.3   8 350.0 245 3.73 3.840 15.41  0  0    3    4
Pontiac Firebird    19.2   8 400.0 175 3.08 3.845 17.05  0  0    3    2
Fiat X1-9           27.3   4  79.0  66 4.08 1.935 18.90  1  1    4    1
Porsche 914-2       26.0   4 120.3  91 4.43 2.140 16.70  0  1    5    2
Lotus Europa        30.4   4  95.1 113 3.77 1.513 16.90  1  1    5    2
Ford Pantera L      15.8   8 351.0 264 4.22 3.170 14.50  0  1    5    4
Ferrari Dino        19.7   6 145.0 175 3.62 2.770 15.50  0  1    5    6
Maserati Bora       15.0   8 301.0 335 3.54 3.570 14.60  0  1    5    8
Volvo 142E          21.4   4 121.0 109 4.11 2.780 18.60  1  1    4    2

    \end{Verbatim}

    \begin{Verbatim}[commandchars=\\\{\}]
{\color{incolor}In [{\color{incolor}10}]:} ULmtcars \PY{o}{\PYZlt{}\PYZhy{}} \PY{k+kp}{unlist}\PY{p}{(}mtcars\PY{p}{)} \PY{c+c1}{\PYZsh{} produces a vector from the dataframe}
         \PY{c+c1}{\PYZsh{} the atomic type of a dataframe is a list}
         \PY{k+kp}{print}\PY{p}{(}ULmtcars\PY{p}{)}
\end{Verbatim}


    \begin{Verbatim}[commandchars=\\\{\}]
   mpg1    mpg2    mpg3    mpg4    mpg5    mpg6    mpg7    mpg8    mpg9   mpg10 
 21.000  21.000  22.800  21.400  18.700  18.100  14.300  24.400  22.800  19.200 
  mpg11   mpg12   mpg13   mpg14   mpg15   mpg16   mpg17   mpg18   mpg19   mpg20 
 17.800  16.400  17.300  15.200  10.400  10.400  14.700  32.400  30.400  33.900 
  mpg21   mpg22   mpg23   mpg24   mpg25   mpg26   mpg27   mpg28   mpg29   mpg30 
 21.500  15.500  15.200  13.300  19.200  27.300  26.000  30.400  15.800  19.700 
  mpg31   mpg32    cyl1    cyl2    cyl3    cyl4    cyl5    cyl6    cyl7    cyl8 
 15.000  21.400   6.000   6.000   4.000   6.000   8.000   6.000   8.000   4.000 
   cyl9   cyl10   cyl11   cyl12   cyl13   cyl14   cyl15   cyl16   cyl17   cyl18 
  4.000   6.000   6.000   8.000   8.000   8.000   8.000   8.000   8.000   4.000 
  cyl19   cyl20   cyl21   cyl22   cyl23   cyl24   cyl25   cyl26   cyl27   cyl28 
  4.000   4.000   4.000   8.000   8.000   8.000   8.000   4.000   4.000   4.000 
  cyl29   cyl30   cyl31   cyl32   disp1   disp2   disp3   disp4   disp5   disp6 
  8.000   6.000   8.000   4.000 160.000 160.000 108.000 258.000 360.000 225.000 
  disp7   disp8   disp9  disp10  disp11  disp12  disp13  disp14  disp15  disp16 
360.000 146.700 140.800 167.600 167.600 275.800 275.800 275.800 472.000 460.000 
 disp17  disp18  disp19  disp20  disp21  disp22  disp23  disp24  disp25  disp26 
440.000  78.700  75.700  71.100 120.100 318.000 304.000 350.000 400.000  79.000 
 disp27  disp28  disp29  disp30  disp31  disp32     hp1     hp2     hp3     hp4 
120.300  95.100 351.000 145.000 301.000 121.000 110.000 110.000  93.000 110.000 
    hp5     hp6     hp7     hp8     hp9    hp10    hp11    hp12    hp13    hp14 
175.000 105.000 245.000  62.000  95.000 123.000 123.000 180.000 180.000 180.000 
   hp15    hp16    hp17    hp18    hp19    hp20    hp21    hp22    hp23    hp24 
205.000 215.000 230.000  66.000  52.000  65.000  97.000 150.000 150.000 245.000 
   hp25    hp26    hp27    hp28    hp29    hp30    hp31    hp32   drat1   drat2 
175.000  66.000  91.000 113.000 264.000 175.000 335.000 109.000   3.900   3.900 
  drat3   drat4   drat5   drat6   drat7   drat8   drat9  drat10  drat11  drat12 
  3.850   3.080   3.150   2.760   3.210   3.690   3.920   3.920   3.920   3.070 
 drat13  drat14  drat15  drat16  drat17  drat18  drat19  drat20  drat21  drat22 
  3.070   3.070   2.930   3.000   3.230   4.080   4.930   4.220   3.700   2.760 
 drat23  drat24  drat25  drat26  drat27  drat28  drat29  drat30  drat31  drat32 
  3.150   3.730   3.080   4.080   4.430   3.770   4.220   3.620   3.540   4.110 
    wt1     wt2     wt3     wt4     wt5     wt6     wt7     wt8     wt9    wt10 
  2.620   2.875   2.320   3.215   3.440   3.460   3.570   3.190   3.150   3.440 
   wt11    wt12    wt13    wt14    wt15    wt16    wt17    wt18    wt19    wt20 
  3.440   4.070   3.730   3.780   5.250   5.424   5.345   2.200   1.615   1.835 
   wt21    wt22    wt23    wt24    wt25    wt26    wt27    wt28    wt29    wt30 
  2.465   3.520   3.435   3.840   3.845   1.935   2.140   1.513   3.170   2.770 
   wt31    wt32   qsec1   qsec2   qsec3   qsec4   qsec5   qsec6   qsec7   qsec8 
  3.570   2.780  16.460  17.020  18.610  19.440  17.020  20.220  15.840  20.000 
  qsec9  qsec10  qsec11  qsec12  qsec13  qsec14  qsec15  qsec16  qsec17  qsec18 
 22.900  18.300  18.900  17.400  17.600  18.000  17.980  17.820  17.420  19.470 
 qsec19  qsec20  qsec21  qsec22  qsec23  qsec24  qsec25  qsec26  qsec27  qsec28 
 18.520  19.900  20.010  16.870  17.300  15.410  17.050  18.900  16.700  16.900 
 qsec29  qsec30  qsec31  qsec32     vs1     vs2     vs3     vs4     vs5     vs6 
 14.500  15.500  14.600  18.600   0.000   0.000   1.000   1.000   0.000   1.000 
    vs7     vs8     vs9    vs10    vs11    vs12    vs13    vs14    vs15    vs16 
  0.000   1.000   1.000   1.000   1.000   0.000   0.000   0.000   0.000   0.000 
   vs17    vs18    vs19    vs20    vs21    vs22    vs23    vs24    vs25    vs26 
  0.000   1.000   1.000   1.000   1.000   0.000   0.000   0.000   0.000   1.000 
   vs27    vs28    vs29    vs30    vs31    vs32     am1     am2     am3     am4 
  0.000   1.000   0.000   0.000   0.000   1.000   1.000   1.000   1.000   0.000 
    am5     am6     am7     am8     am9    am10    am11    am12    am13    am14 
  0.000   0.000   0.000   0.000   0.000   0.000   0.000   0.000   0.000   0.000 
   am15    am16    am17    am18    am19    am20    am21    am22    am23    am24 
  0.000   0.000   0.000   1.000   1.000   1.000   0.000   0.000   0.000   0.000 
   am25    am26    am27    am28    am29    am30    am31    am32   gear1   gear2 
  0.000   1.000   1.000   1.000   1.000   1.000   1.000   1.000   4.000   4.000 
  gear3   gear4   gear5   gear6   gear7   gear8   gear9  gear10  gear11  gear12 
  4.000   3.000   3.000   3.000   3.000   4.000   4.000   4.000   4.000   3.000 
 gear13  gear14  gear15  gear16  gear17  gear18  gear19  gear20  gear21  gear22 
  3.000   3.000   3.000   3.000   3.000   4.000   4.000   4.000   3.000   3.000 
 gear23  gear24  gear25  gear26  gear27  gear28  gear29  gear30  gear31  gear32 
  3.000   3.000   3.000   4.000   5.000   5.000   5.000   5.000   5.000   4.000 
  carb1   carb2   carb3   carb4   carb5   carb6   carb7   carb8   carb9  carb10 
  4.000   4.000   1.000   1.000   2.000   1.000   4.000   2.000   2.000   4.000 
 carb11  carb12  carb13  carb14  carb15  carb16  carb17  carb18  carb19  carb20 
  4.000   3.000   3.000   3.000   4.000   4.000   4.000   1.000   2.000   1.000 
 carb21  carb22  carb23  carb24  carb25  carb26  carb27  carb28  carb29  carb30 
  1.000   2.000   2.000   4.000   2.000   1.000   2.000   2.000   4.000   6.000 
 carb31  carb32 
  8.000   2.000 

    \end{Verbatim}

    \begin{Verbatim}[commandchars=\\\{\}]
{\color{incolor}In [{\color{incolor}11}]:} UCmtcars \PY{o}{\PYZlt{}\PYZhy{}} \PY{k+kp}{unclass}\PY{p}{(}mtcars\PY{p}{)} \PY{c+c1}{\PYZsh{} removes the class attribute, turning the dataframe into a}
         \PY{c+c1}{\PYZsh{} series of vectors plus any names attributes, same as setting}
         \PY{c+c1}{\PYZsh{} class(mtcars) \PYZlt{}\PYZhy{} NULL}
         \PY{k+kp}{print}\PY{p}{(}UCmtcars\PY{p}{)}
\end{Verbatim}


    \begin{Verbatim}[commandchars=\\\{\}]
\$mpg
 [1] 21.0 21.0 22.8 21.4 18.7 18.1 14.3 24.4 22.8 19.2 17.8 16.4 17.3 15.2 10.4
[16] 10.4 14.7 32.4 30.4 33.9 21.5 15.5 15.2 13.3 19.2 27.3 26.0 30.4 15.8 19.7
[31] 15.0 21.4

\$cyl
 [1] 6 6 4 6 8 6 8 4 4 6 6 8 8 8 8 8 8 4 4 4 4 8 8 8 8 4 4 4 8 6 8 4

\$disp
 [1] 160.0 160.0 108.0 258.0 360.0 225.0 360.0 146.7 140.8 167.6 167.6 275.8
[13] 275.8 275.8 472.0 460.0 440.0  78.7  75.7  71.1 120.1 318.0 304.0 350.0
[25] 400.0  79.0 120.3  95.1 351.0 145.0 301.0 121.0

\$hp
 [1] 110 110  93 110 175 105 245  62  95 123 123 180 180 180 205 215 230  66  52
[20]  65  97 150 150 245 175  66  91 113 264 175 335 109

\$drat
 [1] 3.90 3.90 3.85 3.08 3.15 2.76 3.21 3.69 3.92 3.92 3.92 3.07 3.07 3.07 2.93
[16] 3.00 3.23 4.08 4.93 4.22 3.70 2.76 3.15 3.73 3.08 4.08 4.43 3.77 4.22 3.62
[31] 3.54 4.11

\$wt
 [1] 2.620 2.875 2.320 3.215 3.440 3.460 3.570 3.190 3.150 3.440 3.440 4.070
[13] 3.730 3.780 5.250 5.424 5.345 2.200 1.615 1.835 2.465 3.520 3.435 3.840
[25] 3.845 1.935 2.140 1.513 3.170 2.770 3.570 2.780

\$qsec
 [1] 16.46 17.02 18.61 19.44 17.02 20.22 15.84 20.00 22.90 18.30 18.90 17.40
[13] 17.60 18.00 17.98 17.82 17.42 19.47 18.52 19.90 20.01 16.87 17.30 15.41
[25] 17.05 18.90 16.70 16.90 14.50 15.50 14.60 18.60

\$vs
 [1] 0 0 1 1 0 1 0 1 1 1 1 0 0 0 0 0 0 1 1 1 1 0 0 0 0 1 0 1 0 0 0 1

\$am
 [1] 1 1 1 0 0 0 0 0 0 0 0 0 0 0 0 0 0 1 1 1 0 0 0 0 0 1 1 1 1 1 1 1

\$gear
 [1] 4 4 4 3 3 3 3 4 4 4 4 3 3 3 3 3 3 4 4 4 3 3 3 3 3 4 5 5 5 5 5 4

\$carb
 [1] 4 4 1 1 2 1 4 2 2 4 4 3 3 3 4 4 4 1 2 1 1 2 2 4 2 1 2 2 4 6 8 2

attr(,"row.names")
 [1] "Mazda RX4"           "Mazda RX4 Wag"       "Datsun 710"         
 [4] "Hornet 4 Drive"      "Hornet Sportabout"   "Valiant"            
 [7] "Duster 360"          "Merc 240D"           "Merc 230"           
[10] "Merc 280"            "Merc 280C"           "Merc 450SE"         
[13] "Merc 450SL"          "Merc 450SLC"         "Cadillac Fleetwood" 
[16] "Lincoln Continental" "Chrysler Imperial"   "Fiat 128"           
[19] "Honda Civic"         "Toyota Corolla"      "Toyota Corona"      
[22] "Dodge Challenger"    "AMC Javelin"         "Camaro Z28"         
[25] "Pontiac Firebird"    "Fiat X1-9"           "Porsche 914-2"      
[28] "Lotus Europa"        "Ford Pantera L"      "Ferrari Dino"       
[31] "Maserati Bora"       "Volvo 142E"         

    \end{Verbatim}

    \begin{Verbatim}[commandchars=\\\{\}]
{\color{incolor}In [{\color{incolor}12}]:} \PY{k+kt}{c}\PY{p}{(}mtcars\PY{p}{)} \PY{c+c1}{\PYZsh{} similar to unclass but without the attributes}
\end{Verbatim}


    \begin{description}
\item[\$mpg] \begin{enumerate*}
\item 21
\item 21
\item 22.8
\item 21.4
\item 18.7
\item 18.1
\item 14.3
\item 24.4
\item 22.8
\item 19.2
\item 17.8
\item 16.4
\item 17.3
\item 15.2
\item 10.4
\item 10.4
\item 14.7
\item 32.4
\item 30.4
\item 33.9
\item 21.5
\item 15.5
\item 15.2
\item 13.3
\item 19.2
\item 27.3
\item 26
\item 30.4
\item 15.8
\item 19.7
\item 15
\item 21.4
\end{enumerate*}

\item[\$cyl] \begin{enumerate*}
\item 6
\item 6
\item 4
\item 6
\item 8
\item 6
\item 8
\item 4
\item 4
\item 6
\item 6
\item 8
\item 8
\item 8
\item 8
\item 8
\item 8
\item 4
\item 4
\item 4
\item 4
\item 8
\item 8
\item 8
\item 8
\item 4
\item 4
\item 4
\item 8
\item 6
\item 8
\item 4
\end{enumerate*}

\item[\$disp] \begin{enumerate*}
\item 160
\item 160
\item 108
\item 258
\item 360
\item 225
\item 360
\item 146.7
\item 140.8
\item 167.6
\item 167.6
\item 275.8
\item 275.8
\item 275.8
\item 472
\item 460
\item 440
\item 78.7
\item 75.7
\item 71.1
\item 120.1
\item 318
\item 304
\item 350
\item 400
\item 79
\item 120.3
\item 95.1
\item 351
\item 145
\item 301
\item 121
\end{enumerate*}

\item[\$hp] \begin{enumerate*}
\item 110
\item 110
\item 93
\item 110
\item 175
\item 105
\item 245
\item 62
\item 95
\item 123
\item 123
\item 180
\item 180
\item 180
\item 205
\item 215
\item 230
\item 66
\item 52
\item 65
\item 97
\item 150
\item 150
\item 245
\item 175
\item 66
\item 91
\item 113
\item 264
\item 175
\item 335
\item 109
\end{enumerate*}

\item[\$drat] \begin{enumerate*}
\item 3.9
\item 3.9
\item 3.85
\item 3.08
\item 3.15
\item 2.76
\item 3.21
\item 3.69
\item 3.92
\item 3.92
\item 3.92
\item 3.07
\item 3.07
\item 3.07
\item 2.93
\item 3
\item 3.23
\item 4.08
\item 4.93
\item 4.22
\item 3.7
\item 2.76
\item 3.15
\item 3.73
\item 3.08
\item 4.08
\item 4.43
\item 3.77
\item 4.22
\item 3.62
\item 3.54
\item 4.11
\end{enumerate*}

\item[\$wt] \begin{enumerate*}
\item 2.62
\item 2.875
\item 2.32
\item 3.215
\item 3.44
\item 3.46
\item 3.57
\item 3.19
\item 3.15
\item 3.44
\item 3.44
\item 4.07
\item 3.73
\item 3.78
\item 5.25
\item 5.424
\item 5.345
\item 2.2
\item 1.615
\item 1.835
\item 2.465
\item 3.52
\item 3.435
\item 3.84
\item 3.845
\item 1.935
\item 2.14
\item 1.513
\item 3.17
\item 2.77
\item 3.57
\item 2.78
\end{enumerate*}

\item[\$qsec] \begin{enumerate*}
\item 16.46
\item 17.02
\item 18.61
\item 19.44
\item 17.02
\item 20.22
\item 15.84
\item 20
\item 22.9
\item 18.3
\item 18.9
\item 17.4
\item 17.6
\item 18
\item 17.98
\item 17.82
\item 17.42
\item 19.47
\item 18.52
\item 19.9
\item 20.01
\item 16.87
\item 17.3
\item 15.41
\item 17.05
\item 18.9
\item 16.7
\item 16.9
\item 14.5
\item 15.5
\item 14.6
\item 18.6
\end{enumerate*}

\item[\$vs] \begin{enumerate*}
\item 0
\item 0
\item 1
\item 1
\item 0
\item 1
\item 0
\item 1
\item 1
\item 1
\item 1
\item 0
\item 0
\item 0
\item 0
\item 0
\item 0
\item 1
\item 1
\item 1
\item 1
\item 0
\item 0
\item 0
\item 0
\item 1
\item 0
\item 1
\item 0
\item 0
\item 0
\item 1
\end{enumerate*}

\item[\$am] \begin{enumerate*}
\item 1
\item 1
\item 1
\item 0
\item 0
\item 0
\item 0
\item 0
\item 0
\item 0
\item 0
\item 0
\item 0
\item 0
\item 0
\item 0
\item 0
\item 1
\item 1
\item 1
\item 0
\item 0
\item 0
\item 0
\item 0
\item 1
\item 1
\item 1
\item 1
\item 1
\item 1
\item 1
\end{enumerate*}

\item[\$gear] \begin{enumerate*}
\item 4
\item 4
\item 4
\item 3
\item 3
\item 3
\item 3
\item 4
\item 4
\item 4
\item 4
\item 3
\item 3
\item 3
\item 3
\item 3
\item 3
\item 4
\item 4
\item 4
\item 3
\item 3
\item 3
\item 3
\item 3
\item 4
\item 5
\item 5
\item 5
\item 5
\item 5
\item 4
\end{enumerate*}

\item[\$carb] \begin{enumerate*}
\item 4
\item 4
\item 1
\item 1
\item 2
\item 1
\item 4
\item 2
\item 2
\item 4
\item 4
\item 3
\item 3
\item 3
\item 4
\item 4
\item 4
\item 1
\item 2
\item 1
\item 1
\item 2
\item 2
\item 4
\item 2
\item 1
\item 2
\item 2
\item 4
\item 6
\item 8
\item 2
\end{enumerate*}

\end{description}


    
    {References}\\
http://eriqande.github.io/rep-res-web/\\
http://www.pitt.edu/\textasciitilde{}njc23/\\
http://adv-r.had.co.nz/Data-structures.html\\
https://cran.r-project.org/doc/contrib/de\_Jonge+van\_der\_Loo-Introduction\_to\_data\_cleaning\_with\_R.pdf


    % Add a bibliography block to the postdoc
    
    
    
    \end{document}
